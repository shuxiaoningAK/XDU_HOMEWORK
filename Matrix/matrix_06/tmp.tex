\documentclass[11pt,letterpaper]{ctexart}
\textwidth 6.5in
\textheight 9.in
\oddsidemargin 0in
\headheight 0in
\usepackage{graphicx}
\usepackage{fancybox}
\usepackage[utf8]{inputenc} %solucion del problema de los acentos.
\usepackage{epsfig,graphicx}
\usepackage{multicol,pst-plot}
\usepackage{pstricks}
\usepackage{amsmath}
\usepackage{amsfonts}
\usepackage{amssymb}
\usepackage{eucal}
\usepackage[left=2cm,right=2cm,top=2cm,bottom=2cm]{geometry}
\pagestyle{empty}
\DeclareMathOperator{\tr}{Tr}
\newcommand*{\op}[1]{\check{\mathbf#1}}
\newcommand{\bra}[1]{\langle #1 |}
\newcommand{\ket}[1]{| #1 \rangle}
\newcommand{\braket}[2]{\langle #1 | #2 \rangle}
\newcommand{\mean}[1]{\langle #1 \rangle}
\newcommand{\opvec}[1]{\check{\vec #1}}
\renewcommand{\sp}[1]{$${\begin{split}#1\end{split}}$$}

\usepackage{lipsum}

\usepackage{listings}
\usepackage{color}

\definecolor{codegreen}{rgb}{0,0.6,0}
\definecolor{codegray}{rgb}{0.5,0.5,0.5}
\definecolor{codepurple}{rgb}{0.58,0,0.82}
\definecolor{backcolour}{rgb}{0.95,0.95,0.92}

\lstdefinestyle{mystyle}{
	backgroundcolor=\color{backcolour},   
	commentstyle=\color{codegreen},
	keywordstyle=\color{magenta},
	numberstyle=\tiny\color{codegray},
	stringstyle=\color{codepurple},
	basicstyle=\footnotesize,
	breakatwhitespace=false,         
	breaklines=true,                 
	captionpos=b,                    
	keepspaces=true,                 
	numbers=left,                    
	numbersep=5pt,                  
	showspaces=false,                
	showstringspaces=false,
	showtabs=false,                  
	tabsize=2
}

\lstset{style=mystyle}

\begin{document}
\pagestyle{plain}
\begin{flushleft}
ID 22031212122 \\
NAME xiaoning Shu\\
\end{flushleft}

\begin{flushright}\vspace{-18mm}
\includegraphics[height=2.0cm]{logo.png}
\end{flushright}
 
\begin{center}\vspace{-0.1cm}
\textbf{ \large THEORY OF MATRIX}\\
\end{center}

 
\rule{\linewidth}{0.1mm}
%%%%%%%%%%%%%%%%%%%%%%%%%%%%%%%%%%%%%%%%%%%%%%%%%%%%%%%%%%%%%%%%%%%%%%%%

\bigskip
\textbf{\large{Homework}}
\\

{11}



\bigskip
\textbf{\large{Problems}}

\begin{enumerate}

%%%%%%%%%%%%%%%%%%%%
\item Problem 11%%%
%%%%%%%%%%%%%%%%%%%%

对于下列矩阵$A$,求正交(酉)矩阵$P$,使$P^{-1}AP$为对角矩阵:
\begin{enumerate}
	\item $A = \begin{bmatrix}
		2 & 2 & -2 \\
		2 & 5 & -4 \\
		-2 & -4 & 5
	\end{bmatrix}$

	\item $A = \begin{bmatrix}
		0 & j & 1 \\
		-j & 0 & 0 \\
		1 & 0 & 0
	\end{bmatrix}$
\end{enumerate}

\bigskip
\textbf{\large{Answer}}
\begin{enumerate}
	\item  $\lambda I - A = \begin{vmatrix}
		\lambda - 2 & -2 & 2 \\
		-2 & \lambda - 5 & 4 \\
		2 & 4 & \lambda - 5
	\end{vmatrix} = (\lambda - 1)^{2}(\lambda - 10) = 0$.
	
	所以A的特征值$\lambda_1 = \lambda_2 = 1, \lambda_3 = 10$ 

	$\lambda_1 = \lambda_2 = 1$对应的特征向量为$x_1 = (2, 0, 1)^T, x_2 = (-2, 1, 0)^T. \lambda_3$对应的特征值为$x_3 = (1, 2, -2)^T$

	因为有特征值具有2重根,所以对象的向量不正交,将其进行正交化后进行规范化.
	
	采用施密特正交化可得正交矩阵后进行单位化:

	$$ P = \begin{bmatrix}
		\frac{2}{3\sqrt{5}} & -\frac{2}{\sqrt{5}} & -\frac{1}{3} & \\
		\frac{4}{3\sqrt{5}} & \frac{1}{\sqrt{5}} & -\frac{2}{3} & \\
		\frac{5}{3\sqrt{5}} & 0 & \frac{2}{3}
	\end{bmatrix}$$使得$P^{-1}AP = \begin{bmatrix}
		1 & 0 & 0 \\
		0 & 1 & 0 \\
		0 & 0 & 10
	\end{bmatrix}$

	\item $\lambda I - A = \begin{vmatrix}
		\lambda & -j & -1 \\
		j & \lambda & 0 \\
		-1 & 0 & \lambda
	\end{vmatrix} = \lambda(\lambda^2 - 2) = 0$.

	所以A的特征值$\lambda_1 = - \sqrt{2}, \lambda_2 = \sqrt{2}, \lambda_3 = 0$

	$\lambda_1 = -\sqrt{2}, x_1 = (-\sqrt{2}, -j,1)^T,\lambda_2 = \sqrt{2}, x_2 = (\sqrt{2}, -j, 1). \lambda_3 = 0, x_3 = (0, j, 1)$

	因为特征值互不相同,所以特征向量相互正交,只需将其规范化即可.

	所以正交矩阵

	$$ P = \begin{bmatrix}
		-\frac{1}{\sqrt{2}} & \frac{1}{\sqrt{2}} & 0 & \\
		-\frac{j}{2} & -\frac{j}{2} & \frac{j}{\sqrt{2}} & \\
		\frac{1}{2} & \frac{1}{2} & \frac{1}{\sqrt{2}}
	\end{bmatrix}$$使得$P^{-1}AP = \begin{bmatrix}
		-\sqrt{2} & 0 & 0 \\
		0 & \sqrt{2} & 0 \\
		0 & 0 & 0
	\end{bmatrix}$
	
\end{enumerate}











\end{enumerate}
\end{document}

