\documentclass[11pt,letterpaper]{ctexart}
\textwidth 6.5in
\textheight 9.in
\oddsidemargin 0in
\headheight 0in
\usepackage{graphicx}
\usepackage{fancybox}
\usepackage[utf8]{inputenc} %solucion del problema de los acentos.
\usepackage{epsfig,graphicx}
\usepackage{multicol,pst-plot}
\usepackage{pstricks}
\usepackage{amsmath}
\usepackage{amsfonts}
\usepackage{amssymb}
\usepackage{eucal}
\usepackage[left=2cm,right=2cm,top=2cm,bottom=2cm]{geometry}
\pagestyle{empty}
\DeclareMathOperator{\tr}{Tr}
\newcommand*{\op}[1]{\check{\mathbf#1}}
\newcommand{\bra}[1]{\langle #1 |}
\newcommand{\ket}[1]{| #1 \rangle}
\newcommand{\braket}[2]{\langle #1 | #2 \rangle}
\newcommand{\mean}[1]{\langle #1 \rangle}
\newcommand{\opvec}[1]{\check{\vec #1}}
\renewcommand{\sp}[1]{$${\begin{split}#1\end{split}}$$}

\usepackage{lipsum}

\usepackage{listings}
\usepackage{color}

\definecolor{codegreen}{rgb}{0,0.6,0}
\definecolor{codegray}{rgb}{0.5,0.5,0.5}
\definecolor{codepurple}{rgb}{0.58,0,0.82}
\definecolor{backcolour}{rgb}{0.95,0.95,0.92}

\lstdefinestyle{mystyle}{
	backgroundcolor=\color{backcolour},   
	commentstyle=\color{codegreen},
	keywordstyle=\color{magenta},
	numberstyle=\tiny\color{codegray},
	stringstyle=\color{codepurple},
	basicstyle=\footnotesize,
	breakatwhitespace=false,         
	breaklines=true,                 
	captionpos=b,                    
	keepspaces=true,                 
	numbers=left,                    
	numbersep=5pt,                  
	showspaces=false,                
	showstringspaces=false,
	showtabs=false,                  
	tabsize=2
}

\lstset{style=mystyle}

\begin{document}
\pagestyle{plain}
\begin{flushleft}
ID 22031212122 \\
NAME xiaoning Shu\\
\end{flushleft}

\begin{flushright}\vspace{-18mm}
\includegraphics[height=2.0cm]{logo.png}
\end{flushright}
 
\begin{center}\vspace{-0.1cm}
\textbf{ \large THEORY OF MATRIX}\\
\end{center}

 
\rule{\linewidth}{0.1mm}
%%%%%%%%%%%%%%%%%%%%%%%%%%%%%%%%%%%%%%%%%%%%%%%%%%%%%%%%%%%%%%%%%%%%%%%%

\bigskip
\textbf{\large{Homework}}
\\

{3, 4}



\bigskip
\textbf{\large{Problems}}

\begin{enumerate}

%%%%%%%%%%%%%%%%%%%%
\item Problem 3%%%
%%%%%%%%%%%%%%%%%%%%

若$A$为实反对称矩阵$(A^T = -A),$则$e^A$为正交矩阵

\bigskip
\textbf{\large{Answer}}

	由$P152$页推论

	可知$(e^A)^T = e^{A^T} = e^{-A}$

	于是$e^A(e^A)^T = e^Ae^{A^T} = e^{A - A} = e^0 = I$

	所以$e^A$是正交矩阵。



%%%%%%%%%%%%%%%%%%%%
\item Problem 4%%%
%%%%%%%%%%%%%%%%%%%%

若$A$是Hermite矩阵,则$e^{jA}$是酉矩阵

\bigskip
\textbf{\large{Answer}}

	由$Problem 3$可知 $(e^{iA})^H = (e^{iA})^H = e^{-iA}$

	于是有$e^{iA}(e^{iA})^H = e^{iA}e^{-iA} = e^0 = I$

	所以$e^{iA}$是酉矩阵。






\end{enumerate}
\end{document}

