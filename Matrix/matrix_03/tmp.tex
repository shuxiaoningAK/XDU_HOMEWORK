\documentclass[11pt,letterpaper]{ctexart}
\textwidth 6.5in
\textheight 9.in
\oddsidemargin 0in
\headheight 0in
\usepackage{graphicx}
\usepackage{fancybox}
\usepackage[utf8]{inputenc} %solucion del problema de los acentos.
\usepackage{epsfig,graphicx}
\usepackage{multicol,pst-plot}
\usepackage{pstricks}
\usepackage{amsmath}
\usepackage{amsfonts}
\usepackage{amssymb}
\usepackage{eucal}
\usepackage[left=2cm,right=2cm,top=2cm,bottom=2cm]{geometry}
\pagestyle{empty}
\DeclareMathOperator{\tr}{Tr}
\newcommand*{\op}[1]{\check{\mathbf#1}}
\newcommand{\bra}[1]{\langle #1 |}
\newcommand{\ket}[1]{| #1 \rangle}
\newcommand{\braket}[2]{\langle #1 | #2 \rangle}
\newcommand{\mean}[1]{\langle #1 \rangle}
\newcommand{\opvec}[1]{\check{\vec #1}}
\renewcommand{\sp}[1]{$${\begin{split}#1\end{split}}$$}

\usepackage{lipsum}

\usepackage{listings}
\usepackage{color}

\definecolor{codegreen}{rgb}{0,0.6,0}
\definecolor{codegray}{rgb}{0.5,0.5,0.5}
\definecolor{codepurple}{rgb}{0.58,0,0.82}
\definecolor{backcolour}{rgb}{0.95,0.95,0.92}

\lstdefinestyle{mystyle}{
	backgroundcolor=\color{backcolour},   
	commentstyle=\color{codegreen},
	keywordstyle=\color{magenta},
	numberstyle=\tiny\color{codegray},
	stringstyle=\color{codepurple},
	basicstyle=\footnotesize,
	breakatwhitespace=false,         
	breaklines=true,                 
	captionpos=b,                    
	keepspaces=true,                 
	numbers=left,                    
	numbersep=5pt,                  
	showspaces=false,                
	showstringspaces=false,
	showtabs=false,                  
	tabsize=2
}

\lstset{style=mystyle}

\begin{document}
\pagestyle{plain}
\begin{flushleft}
ID 22031212122 \\
NAME xiaoning Shu\\
\end{flushleft}

\begin{flushright}\vspace{-18mm}
\includegraphics[height=2.0cm]{logo.png}
\end{flushright}
 
\begin{center}\vspace{-0.1cm}
\textbf{ \large THEORY OF MATRIX}\\
\end{center}

 
\rule{\linewidth}{0.1mm}
%%%%%%%%%%%%%%%%%%%%%%%%%%%%%%%%%%%%%%%%%%%%%%%%%%%%%%%%%%%%%%%%%%%%%%%%

\bigskip
\textbf{\large{Homework}}
\\

{1, 2, 6, 7}



\bigskip
\textbf{\large{Problems}}

\begin{enumerate}

%%%%%%%%%%%%%%%%%%%%
\item Problem 1%%%
%%%%%%%%%%%%%%%%%%%%

判别下列变换中哪些是线性变换:
\begin{enumerate}
	\item 在$R^3$中,设$x = (\xi_1, \xi_2, \xi_3), Tx = ({\xi_1}^2,{\xi_1} + {\xi_2},\xi_3)$;
    \item 在矩阵空间中$R^{n \times n}$中,$Tx = BXC$,这里$B,C$是固定矩阵;
    \item 在线性空间中$P_n$中,$Tf(t) = f(t + 1)$.
\end{enumerate}

\bigskip
\textbf{\large{Answer}}
\begin{enumerate}
	\item 不是线性变换。
	
	因为${T(2x) = (4\xi_1^2, 2\xi_1 + 2\xi_2, 2\xi_3)}$,但是${2T(x) = (2\xi_1^2, 2\xi_1+ 2\xi_2, 2\xi_3)}$。
    \item 是线性变换。
    \item 是线性变换。
\end{enumerate}

%%%%%%%%%%%%%%%%%%%%
\item Problem 2%%%
%%%%%%%%%%%%%%%%%%%%

在$R^2$中,设 $x = (\xi_1, \xi_2)$. 证明$T_1x = (\xi_2. -\xi_1)$与$T_2x = (\xi_1,-\xi_2)$是 $R^2$的两个线性变换,并求$T_1 + T_2, T_1T_2$及$T_2T_1$ 

\bigskip
\textbf{\large{Answer}}

	设${k,l \in R, y = (\alpha_1, \alpha_2)\in R^2}$,$kx + ly = (k\xi_1 + l\alpha_1, k\xi_2 + l\alpha_2)$
	
	${T_1(kx + ly) = (k\xi_2 + l\alpha_2,-k\xi_1 - l\alpha_1) = kT_1(x) + lT_1(y)}$

	所以$T_1$是线性变换,同理$T_2$也是线性变换.

	所以${(T_1 + T_2)x = T_1(x) + T_2(x) = (\xi_2 + \xi_1, -\xi_1 - \xi_2)}$

	${(T_1T_2)x = T_1T_2(x) = (\-xi_2, -\xi_1)}$

	${(T_2T_1)x = T_2T_1(x) = (\xi_2, \xi_1)}$.




%%%%%%%%%%%%%%%%%%%%
\item Problem 6%%%
%%%%%%%%%%%%%%%%%%%%

六个函数

    $x_1 = e^{at} \cos{b} t$, \quad $x_2 = e_{at}\sin{b}t$, \quad $x_3 = t e^{at}\cos{b}t$

    $x_4 = t e^{at} \sin{b} t$, \quad $x_5 = \frac{1}{2} t^2 e^{at} \cos{b} t$, \quad $x_6 =  \frac{1}{2} t^2 e^{at}\sin{b}t$

    的所有实系数线性组合构成实数域R上的一个六维线性空间$V^6 = L(x_1, x_2,x_3,x_4,x_5,x_6)$,求微分变换D在基$x_1,x_2, \cdots, x_6$下的矩阵


\bigskip
\textbf{\large{Answer}}

% \begin{itemize}
% 	\item 
% \end{itemize}
${D_{x_1}} = ae^{at} \cos{b}t - e^{at} b\sin{b} t = ax_1 - bx_2$

${D_{x_2}} = ae^{at} \sin{b}t + e^{at} b\cos{b} t = bx_1 + ax_2$

${D_{x_3}} = e^{at} \cos{b}t + tae^{at} b\cos{b} t - te^{at} b\sin{b} t = x_1 + ax_3 -bx_4$

${D_{x_4}} = e^{at} \sin{b}t - tae^{at} b\cos{b} t + te^{at} b\cos{b} t= x_2 + bx_3 + ax_4$

${D_{x_5}} = te^{at} \cos{b}t + \frac{1}{2}t^2ae^{at} b\sin{b} t - \frac{1}{2}t^2e^{at} b\sin{b} t= x_3 + ax_5 - bx_6$

${D_{x_6}} = te^{at} \sin{b}t + \frac{1}{2}t^2ae^{at} \sin{b} t + + \frac{1}{2}t^2e^{at} b\cos{b} t = x_4 + bx_5 + ax_6$

所以


$$D = \begin{bmatrix}
    a & b & 1 & 0 & 0 & 0 \\
	-b & a & 0 & 1 & 0 & 0 \\
	0 & 0 & a & b & 1 & 0 \\
	0 & 0 & -b & a & 0 & 1 \\
	0 & 0 & 0 & 0 & a & b \\
	0 & 0 & 0 & 0 & -b & a
\end{bmatrix}$$




%%%%%%%%%%%%%%%%%%%%
\item Problem 7%%%
%%%%%%%%%%%%%%%%%%%%

已知$R^3$的线性变换T在基$x_1 = (-1,1,1), x_2 = (1,0,-1),x_3 = (0,1,1)$下的矩阵是
$$\begin{bmatrix}
    1 & 0 & 1 \\
	1 & 1 & 0 \\
	-1 & 2 & 1
\end{bmatrix}$$
求T在基${e_1 = (1, 0, 0), e_2 = (0, 1, 0), e_3 = (0, 0, 1)}$下的矩阵。

\bigskip
\textbf{\large{Answer}}

由题意得:
$C^{-1} = $
$\begin{bmatrix}
    1 & 0 & 1 \\
	1 & 1 & 0 \\
	-1 & 2 & 1
\end{bmatrix}$

所以在新基下的矩阵为
$C^{-1}
\begin{bmatrix}
    1 & 0 & 1 \\
	1 & 1 & 0 \\
	-1 & 2 & 1
\end{bmatrix}C = 
\begin{bmatrix}
    -1 & 1 & -2 \\
	2 & 2 & 0 \\
	3 & 0 & 2
\end{bmatrix}
$


















\end{enumerate}
\end{document}

