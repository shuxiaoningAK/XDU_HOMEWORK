\documentclass[11pt,letterpaper]{ctexart}
\textwidth 6.5in
\textheight 9.in
\oddsidemargin 0in
\headheight 0in
\usepackage{graphicx}
\usepackage{fancybox}
\usepackage[utf8]{inputenc} %solucion del problema de los acentos.
\usepackage{epsfig,graphicx}
\usepackage{multicol,pst-plot}
\usepackage{pstricks}
\usepackage{amsmath}
\usepackage{amsfonts}
\usepackage{amssymb}
\usepackage{eucal}
\usepackage[left=2cm,right=2cm,top=2cm,bottom=2cm]{geometry}
\pagestyle{empty}
\DeclareMathOperator{\tr}{Tr}
\newcommand*{\op}[1]{\check{\mathbf#1}}
\newcommand{\bra}[1]{\langle #1 |}
\newcommand{\ket}[1]{| #1 \rangle}
\newcommand{\braket}[2]{\langle #1 | #2 \rangle}
\newcommand{\mean}[1]{\langle #1 \rangle}
\newcommand{\opvec}[1]{\check{\vec #1}}
\renewcommand{\sp}[1]{$${\begin{split}#1\end{split}}$$}

\usepackage{lipsum}

\usepackage{listings}
\usepackage{color}

\definecolor{codegreen}{rgb}{0,0.6,0}
\definecolor{codegray}{rgb}{0.5,0.5,0.5}
\definecolor{codepurple}{rgb}{0.58,0,0.82}
\definecolor{backcolour}{rgb}{0.95,0.95,0.92}

\lstdefinestyle{mystyle}{
	backgroundcolor=\color{backcolour},   
	commentstyle=\color{codegreen},
	keywordstyle=\color{magenta},
	numberstyle=\tiny\color{codegray},
	stringstyle=\color{codepurple},
	basicstyle=\footnotesize,
	breakatwhitespace=false,         
	breaklines=true,                 
	captionpos=b,                    
	keepspaces=true,                 
	numbers=left,                    
	numbersep=5pt,                  
	showspaces=false,                
	showstringspaces=false,
	showtabs=false,                  
	tabsize=2
}

\lstset{style=mystyle}

\begin{document}
\pagestyle{plain}
\begin{flushleft}
ID 22031212122 \\
NAME xiaoning Shu\\
\end{flushleft}

\begin{flushright}\vspace{-18mm}
\includegraphics[height=2.0cm]{logo.png}
\end{flushright}
 
\begin{center}\vspace{-0.1cm}
\textbf{ \large THEORY OF MATRIX}\\
\end{center}

 
\rule{\linewidth}{0.1mm}
%%%%%%%%%%%%%%%%%%%%%%%%%%%%%%%%%%%%%%%%%%%%%%%%%%%%%%%%%%%%%%%%%%%%%%%%

\bigskip
\textbf{\large{Homework}}
\\

{5, 6, x}



\bigskip
\textbf{\large{Problems}}

\begin{enumerate}

%%%%%%%%%%%%%%%%%%%%
\item Problem 5%%%
%%%%%%%%%%%%%%%%%%%%

设$A = \begin{bmatrix}
	2 & 1 & 0 \\
	0 & 0 & 1 \\
	0 & 1 & 0
\end{bmatrix}$求$e^A, e^{tA}(t \in R), sinA$

\bigskip
\textbf{\large{Answer}}

	



%%%%%%%%%%%%%%%%%%%%
\item Problem 6%%%
%%%%%%%%%%%%%%%%%%%%

若$f(z) = \ln{z}$,求$f(A)$,这里A为

$(1) A = \begin{bmatrix}
	1 & 0 & 0 & 0 \\
	1 & 1 & 0 & 0 \\
	0 & 1 & 1 & 0 \\
	0 & 0 & 1 & 1
\end{bmatrix}; (2) A = \begin{bmatrix}
	2 & 1 & 0 & 0 \\
	0 & 2 & 0 & 0 \\
	0 & 0 & 1 & 1 \\
	0 & 0 & 0 & 1
\end{bmatrix}$

\bigskip
\textbf{\large{Answer}}



%%%%%%%%%%%%%%%%%%%%
\item Problem x%%%
%%%%%%%%%%%%%%%%%%%%

设矩阵$A = \begin{bmatrix}
	4 & 6 & 0 \\
	-3 & -5 & 0 \\
	-3 & -6 & 1
\end{bmatrix}$求$e^A$.




\end{enumerate}
\end{document}

